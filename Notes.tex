\documentclass[10pt, oneside]{article} 
\usepackage{amsmath, amsthm, amssymb, calrsfs, wasysym, verbatim, bbm, color, graphics, geometry}

\geometry{tmargin=.75in, bmargin=.75in, lmargin=.75in, rmargin = .75in}  

\newcommand{\R}{\mathbb{R}}
\newcommand{\C}{\mathbb{C}}
\newcommand{\Z}{\mathbb{Z}}
\newcommand{\N}{\mathbb{N}}
\newcommand{\Q}{\mathbb{Q}}
\newcommand{\Cdot}{\boldsymbol{\cdot}}

\newtheorem{thm}{Theorem}
\newtheorem{defn}{Definition}
\newtheorem{conv}{Convention}
\newtheorem{rem}{Remark}
\newtheorem{lem}{Lemma}
\newtheorem{cor}{Corollary}
\newtheorem{example}{Example}

\title{Integral Equations Notes: [457]}
\author{[Drew Remmenga]}


\begin{document}

\maketitle

\vspace{.25in}

\begin{defn}
    An integral equation is any equyation in which the unknown function is inside the integral sign.
\end{defn}
\begin{example}
    \begin{align*}
        \int_{a}^{b} k(x,y)u(y)dy = f(x), x \in (a,b)
    \end{align*}
    Here k and f are given and u(x) is the unknown function.
\end{example}
\begin{example}
    \begin{align*}
        \int_{0}^{\infty} e ^{-xt}u(t)dt = f(x), x \in (0,\infty)
    \end{align*}
    f(x) is the laplace transform of u.
\end{example}
\begin{rem}
    There is an inversion formula giving u from f but it involves an integhrral of f(x) over complex values of x. In practice we may only know f(x) for real values of x.
\end{rem}
\begin{example}
    \begin{align*}
        \int_{-1}^{1} u(t) dt = 1 \intertext{One solution} u(t) = \frac{1}{2}, t \in (-1,1)
    \end{align*}
    Clearly there are lots of solutions, e.g.
    \begin{align*}
        u(t) = \frac{1}{2} + \intertext{any odd funtion of t.}
    \end{align*}
    or
    \begin{align*}
        u(t) = \frac{1}{2} + \sum_{n=1}^{\infty} \{ a_{n} cos(nt\pi)+b_{n} sin(nt\pi)\}
    \end{align*}
    For any reasonable $a_{n}$ and $b_{n}$.
\end{example}
\end{document}