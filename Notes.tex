\documentclass[10pt, oneside]{article} 
\usepackage{amsmath, amsthm, amssymb, calrsfs, wasysym, verbatim, bbm, color, graphics, geometry}
\usepackage{mathrsfs}
\geometry{tmargin=.75in, bmargin=.75in, lmargin=.75in, rmargin = .75in}  

\newcommand{\R}{\mathbb{R}}
\newcommand{\C}{\mathbb{C}}
\newcommand{\Z}{\mathbb{Z}}
\newcommand{\N}{\mathbb{N}}
\newcommand{\Q}{\mathbb{Q}}

\newcommand{\Cdot}{\boldsymbol{\cdot}}

\newtheorem{thm}{Theorem}
\newtheorem{defn}{Definition}
\newtheorem{conv}{Convention}
\newtheorem{rem}{Remark}
\newtheorem{lem}{Lemma}
\newtheorem{cor}{Corollary}
\newtheorem{example}{Example}
\newtheorem{exe}{Exercise}

\title{Integral Equations Notes: [457]}
\author{[Drew Remmenga]}


\begin{document}

\maketitle

\vspace{.25in}

\begin{defn}
    An integral equation is any equyation in which the unknown function is inside the integral sign.
\end{defn}
\begin{example}
    \begin{align*}
        \int_{a}^{b} k(x,y)u(y)dy = f(x), x \in (a,b)
    \end{align*}
    Here k and f are given and u(x) is the unknown function.
\end{example}
\begin{example}
    \begin{align*}
        \int_{0}^{\infty} e ^{-xt}u(t)dt = f(x), x \in (0,\infty)
    \end{align*}
    f(x) is the laplace transform of u.
\end{example}
\begin{rem}
    There is an inversion formula giving u from f but it involves an integhrral of f(x) over complex values of x. In practice we may only know f(x) for real values of x.
\end{rem}
\begin{example}
    \begin{align*}
        \int_{-1}^{1} u(t) dt = 1 \intertext{One solution} u(t) = \frac{1}{2}, t \in (-1,1)
    \end{align*}
    Clearly there are lots of solutions, e.g.
    \begin{align*}
        u(t) = \frac{1}{2} + \intertext{any odd funtion of t.}
    \end{align*}
    or
    \begin{align*}
        u(t) = \frac{1}{2} + \sum_{n=1}^{\infty} \{ a_{n} cos(nt\pi)+b_{n} sin(nt\pi)\}
    \end{align*}
    For any reasonable $a_{n}$ and $b_{n}$.
\end{example}
\begin{example}
    \begin{align*}
        \int_{-1}^{1} u(t)dt = f(x), x \in (-1,1)
    \end{align*}
    Now there are no solutions unless f(x) is constant.
\end{example}
\begin{example}
    \begin{align}
        \alpha u(x) + \beta \int_{-1}^{1} u(t) dt = f(x), x \in (-1,1)
    \end{align}
    Where $\alpha$ and $\beta$ are given constants with $\alpha \neq 0$ or example 4 put $\int_{-1}^{1} u(t) dt = c$ a known constant. Then (1) gives
    \begin{align}
        u(x) = \frac{1}{\alpha} \{f(x) - \beta C\}
    \end{align}
    So we will have solved (1) and know C. Integrate (2) gives
    \begin{align*}
        C = \int_{-1}^{1} u(x) dx = \frac{1}{\alpha} \int_{-1}^{1} f(x)dx - \frac{\beta C}{\alpha}\int_{-1}^{1} dx
    \end{align*}
    \begin{align}
        (1+\frac{2\beta}{\alpha})C = \frac{1}{\alpha}\int_{-1}^{1} f(x)dx
    \end{align}
    \begin{align*}
        C = \frac{1}{\alpha + 2\beta}\int_{-1}^{1} f(x) dx
    \end{align*}
    then (2) gives unique solution of (1) real provided $\alpha + 2 \beta \neq 0$. What happens if $\alpha + 2\beta =0 $? As before we ghet to (3) which reduces to 
    \begin{align}
        0\int_{-1}^{1} f(x) dx
    \end{align}
    If f(x) satisfies (4) eg f cpould be odd then (2) solves (1) for any choice of the constant C: we have existence but not uniqueness. If f(x) does not satisfy (4) eg f(x) could by $x^{2}$ then (1) does not have any solutions $\alpha + 2\beta=0$. All this is reminicent of linear algebra (solving $Ax=\underset{\tilde{}}{b}$).
\end{example}
\begin{exe}
    Put $\frac{\beta}{\alpha}=-\lambda$ and $f=0$ giving
    \begin{align*}
        u(x) = \int_{-1}^{1} u(t) dt = 0
    \end{align*}
    A homogenous integral equation. Are there values of $\lambda$ for which this integral equation has non trivial solution $u\neq 0$?
\end{exe}
\begin{example}
    \begin{align}
        \int_{0}^{t} u (\tau) d\tau = f(t), t > 0
    \end{align}
    Differentiate with respect to x(t):
    \begin{align*}
        u(x) = f'(x)
    \end{align*}
    Two anxieties.
    1. (5) suggests that f(0)=0 in $\lim_{x\to 0}$ (5)  what happens if $f(0) \neq 0$ eg $f(x) = 1, \forall x$
    2. What happens if f is not differentiable? Try using laplace transform. Let $U(s) = \mathscr{L}\{u\}$
    \begin{align*}
        \int_{0}^{\infty} u(t) e^{-st} dt 
    \end{align*}
    We have:
    \begin{align*}
        \mathscr{L}\{\int_{0}^{t} u(\tau) d\tau \} &= \int_{t=0}^{\infty} e^{-st}\int_{0}^{\tau = t} u(\tau) d\tau \\
        &= \int_{0}^{\infty}u(\tau) \int_{\tau}^{\infty} e^{-st} dt d\tau\\
        &= \frac{e^{st}}{s}\|_{\tau}^{infty} = \frac{e^{-s\tau}}{s} \\
        &= \int_{0}^{\infty} u(\tau) \frac{e^{-s\tau}}{s} d\tau\\
        &= \frac{1}{s} \int_{0}^{\infty} u(\tau) e^{-s\tau}d\tau\\
        &= \frac{1}{s} \mathscr{L}\{u\}\\
        \mathscr{L} \{ (5) \} &=\frac{1}{s} U(s) = F(s)\\
        & \implies \\
        u(s) & = sF(s)\\
        \mathscr{L} \{ f'\} & = sF(s) -f(0)\\
        \mathscr{L} \{ u \} & = U = sF(s) +sf(x)- f(0)\\
        &= \mathscr{L} \{f\} + f(0) 
    \end{align*}
    Question is is there a g(t) such that $\mathscr{L} \{ g \} =1$
\end{example}
\end{document}