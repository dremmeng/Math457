\documentclass[10pt, oneside]{article} 
\usepackage{amsmath, amsthm, amssymb, calrsfs, wasysym, verbatim, bbm, color, graphics, geometry}
\usepackage{mathrsfs}
\geometry{tmargin=.75in, bmargin=.75in, lmargin=.75in, rmargin = .75in}  

\newcommand{\R}{\mathbb{R}}
\newcommand{\C}{\mathbb{C}}
\newcommand{\Z}{\mathbb{Z}}
\newcommand{\N}{\mathbb{N}}
\newcommand{\Q}{\mathbb{Q}}

\newcommand{\Cdot}{\boldsymbol{\cdot}}

\newtheorem{thm}{Theorem}
\newtheorem{defn}{Definition}
\newtheorem{conv}{Convention}
\newtheorem{rem}{Remark}
\newtheorem{lem}{Lemma}
\newtheorem{cor}{Corollary}
\newtheorem{example}{Example}
\newtheorem{exe}{Exercise}

\title{Integral Equations Notes: [457]}
\author{[Drew Remmenga]}


\begin{document}

\maketitle

\vspace{.25in}

\begin{defn}
    An integral equation is any equyation in which the unknown function is inside the integral sign.
\end{defn}
\begin{example}
    \begin{align*}
        \int_{a}^{b} k(x,y)u(y)dy = f(x), x \in (a,b)
    \end{align*}
    Here k and f are given and u(x) is the unknown function.
\end{example}
\begin{example}
    \begin{align*}
        \int_{0}^{\infty} e ^{-xt}u(t)dt = f(x), x \in (0,\infty)
    \end{align*}
    f(x) is the laplace transform of u.
\end{example}
\begin{rem}
    There is an inversion formula giving u from f but it involves an integhrral of f(x) over complex values of x. In practice we may only know f(x) for real values of x.
\end{rem}
\begin{example}
    \begin{align*}
        \int_{-1}^{1} u(t) dt = 1 \intertext{One solution} u(t) = \frac{1}{2}, t \in (-1,1)
    \end{align*}
    Clearly there are lots of solutions, e.g.
    \begin{align*}
        u(t) = \frac{1}{2} + \intertext{any odd funtion of t.}
    \end{align*}
    or
    \begin{align*}
        u(t) = \frac{1}{2} + \sum_{n=1}^{\infty} \{ a_{n} cos(nt\pi)+b_{n} sin(nt\pi)\}
    \end{align*}
    For any reasonable $a_{n}$ and $b_{n}$.
\end{example}
\begin{example}
    \begin{align*}
        \int_{-1}^{1} u(t)dt = f(x), x \in (-1,1)
    \end{align*}
    Now there are no solutions unless f(x) is constant.
\end{example}
\begin{example}
    \begin{align}
        \alpha u(x) + \beta \int_{-1}^{1} u(t) dt = f(x), x \in (-1,1)
    \end{align}
    Where $\alpha$ and $\beta$ are given constants with $\alpha \neq 0$ or example 4 put $\int_{-1}^{1} u(t) dt = c$ a known constant. Then (1) gives
    \begin{align}
        u(x) = \frac{1}{\alpha} \{f(x) - \beta C\}
    \end{align}
    So we will have solved (1) and know C. Integrate (2) gives
    \begin{align*}
        C = \int_{-1}^{1} u(x) dx = \frac{1}{\alpha} \int_{-1}^{1} f(x)dx - \frac{\beta C}{\alpha}\int_{-1}^{1} dx
    \end{align*}
    \begin{align}
        (1+\frac{2\beta}{\alpha})C = \frac{1}{\alpha}\int_{-1}^{1} f(x)dx
    \end{align}
    \begin{align*}
        C = \frac{1}{\alpha + 2\beta}\int_{-1}^{1} f(x) dx
    \end{align*}
    then (2) gives unique solution of (1) real provided $\alpha + 2 \beta \neq 0$. What happens if $\alpha + 2\beta =0 $? As before we ghet to (3) which reduces to 
    \begin{align}
        0\int_{-1}^{1} f(x) dx
    \end{align}
    If f(x) satisfies (4) eg f cpould be odd then (2) solves (1) for any choice of the constant C: we have existence but not uniqueness. If f(x) does not satisfy (4) eg f(x) could by $x^{2}$ then (1) does not have any solutions $\alpha + 2\beta=0$. All this is reminicent of linear algebra (solving $Ax=\underset{\tilde{}}{b}$).
\end{example}
\begin{exe}
    Put $\frac{\beta}{\alpha}=-\lambda$ and $f=0$ giving
    \begin{align*}
        u(x) = \int_{-1}^{1} u(t) dt = 0
    \end{align*}
    A homogenous integral equation. Are there values of $\lambda$ for which this integral equation has non trivial solution $u\neq 0$?
\end{exe}
\begin{example}
    \begin{align}
        \int_{0}^{t} u (\tau) d\tau = f(t), t > 0
    \end{align}
    Differentiate with respect to x(t):
    \begin{align*}
        u(x) = f'(x)
    \end{align*}
    Two anxieties.
    1. (5) suggests that f(0)=0 in $\lim_{x\to 0}$ (5)  what happens if $f(0) \neq 0$ eg $f(x) = 1, \forall x$
    2. What happens if f is not differentiable? Try using laplace transform. Let $U(s) = \mathscr{L}\{u\}$
    \begin{align*}
        \int_{0}^{\infty} u(t) e^{-st} dt 
    \end{align*}
    We have:
    \begin{align*}
        \mathscr{L}\{\int_{0}^{t} u(\tau) d\tau \} &= \int_{t=0}^{\infty} e^{-st}\int_{0}^{\tau = t} u(\tau) d\tau \\
        &= \int_{0}^{\infty}u(\tau) \int_{\tau}^{\infty} e^{-st} dt d\tau\\
        &= \frac{e^{st}}{s}\|_{\tau}^{infty} = \frac{e^{-s\tau}}{s} \\
        &= \int_{0}^{\infty} u(\tau) \frac{e^{-s\tau}}{s} d\tau\\
        &= \frac{1}{s} \int_{0}^{\infty} u(\tau) e^{-s\tau}d\tau\\
        &= \frac{1}{s} \mathscr{L}\{u\}\\
        \mathscr{L} \{ (5) \} &=\frac{1}{s} U(s) = F(s)\\
        & \implies \\
        u(s) & = sF(s)\\
        \mathscr{L} \{ f'\} & = sF(s) -f(0)\\
        \mathscr{L} \{ u \} & = U = sF(s) +sf(x)- f(0)\\
        &= \mathscr{L} \{f\} + f(0) 
    \end{align*}
    Question is is there a g(t) such that $\mathscr{L} \{ g \} =1$ Answer: It depends on what you mean by a function. 
    \begin{align*}
        LHS = \mathscr{L} \{g\} = \int_{0}^{\infty} g(t) e^{st} dt = G(s)
    \end{align*}
    Fact: If g is an ordinary integrable function then we must not have $G(s) \to 0$ as $s \to \infty$ As we want G(s) = 1. g can't be ordinary. Fact: $\mathscr{L} \{\delta\}=1$ Wwhere $\delta$ = dirac delta function. SO solution of (5).
    \begin{align*}
        u(x) =f'(x) + f(0)dx
    \end{align*} 
    In particular if $f(0) \neq 0$ then (5) has no conbtinuous solutions for $x \geq 0$.
    Second Anxiety: Consider (5) where f is not a continous function. 
\end{example}
\begin{example}
    \begin{align*}
    f(x) &= \{\begin{cases} 0, & 0 \leq x < c \\
        x-c, & x \geq c
    \end{cases}
    & = (x-c)H(x-c)
    \end{align*}
    H(x) is the divided unit function. Then 
    \begin{align*}
        \mathscr{L} \{ (5) \} = \frac{1}{s} u(s) = \mathscr{L}\{ f\} & = \mathscr{L}\{ x-c)H(x-c)\} \\
        & = \int_{0}^{\infty} (x-c)H(x-c) e^{-sx}dx\\
        & = \int_{0}^{\infty} (x-c)e^{sx} dx, (x-c) = y\\
        & = \int_{0} ^{\infty} ye^{ -s(y+c)}dy \\
        & = e^{-sc}\int_{0} ^{\infty} yce^{-sy}dy\\
        & = e^{- sc}\mathscr{L} \{ t\} = \frac{e^{-sc}}{s^{2}} \\
        & \implies U(s) =se^{-sc} \\
        u(x) &= H(x-c)  
    \end{align*}

\end{example}
\begin{example}
    \begin{align*}
        u(x) = \frac{\alpha}{\pi} \int_{-1}^{1} \frac{u(t)dt}{(x-2)^{2}+\alpha^{2}} =1, x\in (-1,1)
    \end{align*}
    The integral contains a parameter $\alpha$. It turns up in at least two distinct physical applications. One in electro statics. whgere it is called Love's equation and in quantum phases where it is called Lieb's equation. It looks simple byt no explicit solution is known. Soi what is known is that u(x) exists, its continuous, it's unique, and it's even. Numerical solutions are easy to compute unless $\alpha$ is very small.  
\end{example}
\section{Abel's Equation}
A bead starts froim rest at P and slides down a smooth wire riching 0 at time $t_{0}$ Find $t_{p}$. Motion is under gravity in a vertical plane put point P at (x,y) and Q at $(\xi,\delta)$. Let s denote the length from 0 to Q. Energy at P = Energy at Q. m = mass of the bead. 
\begin{align*}
    mgy &= mg\delta+\frac{1}{2} m (\frac{ds}{dt})^{2}\\
    \frac{ds}{dt} &= -\sqrt{2gz(y-\delta)}
\end{align*}
integrating 
\begin{align*}
    t_p = \int_{0}^{t_p}dt= \int_{\delta=y}^{\delta=0} \frac{ds}{\sqrt{2g(y-\delta)}} 
\end{align*}
Next specify the shape (equation) of the wire. Its convienient to suppost that $s=S(\delta)$ for some function S. Thus $ds = S(\gamma)d\gamma$
\begin{align*}
    t_p = \int_{y}^{0} \frac{s'(\gamma)d\gamma}{\sqrt{2g(y-\gamma)}} 
\end{align*}
This gives $t_p$ if we know the shape of the wire. Abel asked given $t_p$ can we find S? 
Put $u(\gamma) = S'(\gamma)$ and $f(y) = \sqrt{2g}t_{p}$ giving:
\begin{align}
    \int_{0}^{y}\frac{u(\gamma)d\gamma}{\sqrt{y-\gamma}} = f(y), y>0
\end{align}
This is Abel's integral equation. LHS of (6) is a laplace convolution. Recall:
\begin{align*}
    (f*g)(t) &= \int_{0}^{t}f(\gamma)g(t-\gamma)d\gamma \\
    \mathscr{L}\{f*g\} & = \mathscr{L}\{f\} \mathscr{L}\{g\}
\end{align*}
\begin{proof}
    \begin{align*}
        \mathscr{L}\{f*g\} &= \int_{0}^{\infty}e^{-st} (f*g)(t)dt \\
        & = \int_{0}^{\infty}e^{-st}\int_{0}^{t}f(\gamma)g(t-\gamma)d\gamma dt\\
        & = \int_{0}^{\infty}f(\gamma)\int_{\delta}^{\infty}e^{-st} g(t-\gamma)dtd\gamma\\
        & = \int_{0}^{\infty}f(\gamma)\int_{x=0}^{\infty}e^{-s(\gamma + x)}g(x)dxd\gamma\\
        & = \mathscr{L}\{f\} \mathscr{L}\{g\} 
    \end{align*}
\end{proof}
\begin{align}
    \mathscr{L}\{(6)\}  : U(s) \mathscr{L}\{\frac{1}{\sqrt{t}}\} = F(s)
\end{align}
Now:
\begin{align*}
    \mathscr{L} \{t^{\nu}\} & = \int_{0}^{\infty}t^{\nu}e^{-st}dt \\
    & = \int_{0}^{\infty}(\frac{x}{s})^{\nu}e^{-x} \frac{dx}{s} \\
    \text{Put } st & = x \\
    & = \frac{1}{st^{\nu+1}} \int_{0}^{\infty}x^{\nu}e^{-x} dx, \nu > -1\\
\end{align*}
\end{document}